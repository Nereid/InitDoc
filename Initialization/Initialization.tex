\documentclass[a4paper]{article}

\usepackage{hyperref}
\hypersetup{
colorlinks=false,
pdfborder={0 0 0}
}
\usepackage[utf8]{inputenc}
\usepackage[francais,english]{babel}
\usepackage[top=2cm, bottom=2cm, left=2cm, right=2cm]{geometry}
\usepackage{graphicx}
\usepackage[final]{pdfpages}
\usepackage{rotating}
\usepackage{eurosym}
\usepackage{lscape}
\usepackage{float}
\usepackage{color}
\usepackage{colortbl}

% Personal macros (or libraries)
% \input{include.tex}

\begin{document}
\title{Web support for Academic Exchange Students : Initialization}
\author{Martin RICHARD and Al.}

%------------------------------------- Page de titre
\maketitle
\abstract{This document will summary the project management plan as seen at the
	beginning of the project. It will describe the project, including the
	deliverables that are expected. The document also contains a global project
	planning, a general description the teammates' roles and a description of the
	tools we will use. This document stand for a reference guide and therefore can
	be updated and enhanced during the project, in order to fill blanks over
	time.}

%\begin{titlepage}
%~

%\vfill
%\begin{Large}
%Septembre 2011
%\end{Large}
%\vfill
%\end{titlepage}
%----------------------------------------------------

%--------------------------------- Table des matières
\newpage
\tableofcontents
\newpage
%----------------------------------------------- Plan

\section{Introduction}

\subsection{Project summary}

\paragraph{} At the initiative the Mme Rumpler and the IT department from the
INSA, we would like to design and build a website providing support for exchange
students, including features allowing older students to mentor the ones that are
about to move or currently abroad.

\subsection{Context of the project}

\paragraph{} This project is a scholar assignment realized by a team of
students from France and Germany. At the very least, this project had been set
with the intention of making students discover the specificities of a project on
an international scale. Such a project involves teammates working together from
various locations and speaking different languages. It will obviously imply that
the project organization is adapted to this very specific situation.
Therefore, we must take a special care of a good communication channel between
the two teams.

\paragraph{} This project started on week 42 by kick-off meetings with tutoring
teachers in both schools. The project we are working on is pushed by Mme
Rumpler and the French School (INSA de Lyon). Each stage of the project must be
validated by Mme Rumpler in order to guarantee that the project requirements
are well understood and followed.

\paragraph{} The project agenda sets the last review (and therefore the
deadline) to the second week of January '12. We should have enough time to
build a real-world application that meets the project requirements.
Nevertheless, miscommunication between the two parts of the team is a major risk
of failure.

\newpage
\section{Description}

\subsection{Short description}
\paragraph{} We would like to create a website offering guides, tips and
information about student exchanges. The website will provide data sorted by
countries and universities.

\paragraph{} We hope to be able to turn the webiste online by the end of the
year, and provide a stable version during January.

\paragraph{} Not much details about the structure of the website are provided.
Our main task by now will probably to write down a document describing how we
would like to present the website.

\paragraph{} The website will be fed by students.

\paragraph{} Access to the content of the website must be restricted to
students. Further details will be provided in the technical subsection.

\subsection{Technical requirements}

\paragraph{} We must host the website on a server provided by our local student
club. It will probably be a classic LAMP stack (Linux/Apache/Mysql/PHP). We will
not be able to choose to work with an other langage (Ruby, python, ...).

\paragraph{} A Centralized Authentication System (CAS) had been deployed on our
campus. CAS is a standard technology allowing to authenticate user on many
web services using one centralized couple login/password by user. We must use
this platform to grant access to users on the website.

\subsection{Recomandations}

\paragraph{} It has been explicitely asked not to implement a rating or reviews
system, as the teachers don't want the website to contain negative comments from
students.


\newpage
\section{Forecast planning}

\paragraph{} The following forecast planning is presented in order to describe
the project execution. This planning will change and become more accurate over
the time. This planning doesn't describe tasks but work packages. A forecast
planning with too much detail shall be thrown away because the lack of details
on the content of the final product.

\paragraph{} The project will last 10 weeks, including the two weeks of
Christmas Holidays in France. This planning starts at week 44.

\paragraph{} The following list is an estimation of the duration of each project
steps, taking into account a student agenda.

\begin{itemize}
	\item Project set-up: 1 week
	\item Website specifications: 2 week
	\item CMSs benchmarks and tests: 0.5 week
	\item Set-up of the development environment: 0.5 week
	\item Construction (Integration, code and tests): 3 weeks
	\item Users tests and validation: 1 week
	\item Release: 0 week (End of the project)
	\item Total: 8 weeks
\end{itemize}

\paragraph{} I think that holidays are meant for a good reason, we shall not
need to work too much during the last two weeks of the year if we're not too
late.

\paragraph{} The following list gives a hint about expected deadlines (consider
the end of the day). Dates are in format Month/Day.

\begin{itemize}
	\item Project set-up: Friday 11/4
	\item Website specifications: 11/22
	\item CMSs benchmarks and tests: 11/22
	\item Set-up of the development environment: 11/25
	\item Construction (Integration, code and tests): 12/17
	\item Users tests and validation: 1/4
	\item Release: 1/9
\end{itemize}

\paragraph{} As you may have noticed, write website specifications and
benchmarking CMSs share the same deadline. These two activities must be
performed simultaneously. User tests and validation requires the team to collect
feedback from other students and test the website. I think we can do it during
the holidays, in order to preserve a few days at the end of the project to
prepare the final presentation or work on various details.

\newpage
\section{Teamates' roles}

\paragraph{} While trying to make a clear list of our respective roles, it
appeared that at the exception of PM and QM, all the teammates will switch from a
role to an other from a step to an other.

\subsection{Project Manager}

\paragraph{} The role of project manager is known of all of us, for the sake of
efficiency, I won't write more about this topic.

\subsection{Quality Manager}

\paragraph{} The quality manager will prepare validation procedures and tools
that we will follow while working on the project and validate the work that is
done according to these procedures.

\paragraph{} Quality Manager work includes listing of points of interest to
check for each step of the project, sometimes document templates and will
prepare tests and Quality Assurance feedback tools.

\subsection{Architect/designer}

\paragraph{} This role includes various tasks. From both technical and
functional side, designers will sketch the website and describe its features.

\paragraph{} The designer will write out the project requirements.

\paragraph{} In order to succeed in its work, the designer will list
best-practices to follow, compare existing websites and technologies that are
meaningful for the project.

\subsection{Developer}

\paragraph{} In this context, the developer will not exactly write a lot of
code. The first job (and not the easiest) of the developer is to discover and
understand the stack of tools the project rely on : from the browser, on client
side to the code framework and APIs from the CMS we'll use, on server side.

\paragraph{} The developer main task will be to set-up and integrate plug-ins
and "modules" to the chosen framework or CMS. The selection of a module depends
on the features it provides, the quality of it's code and it's compatibility
with other modules. The developer will write code for modules when no existing
one statisfies the requirements or has to be updated.

\subsection{Webdesginer/Front-end developer}

\paragraph{} The webdeisgner/Front developer focuses his work on the design and
aspect of the website. We will probably start from an existing template, so the
work of the webdesigner will be to adjust it to our needs.

\paragraph{} This job requires technical skills in HTML/CSS technologies more
than an artistic talent or Photoshop skills.

\newpage
\section{Management methods}

\paragraph{} Classical management method don't fit well with a web project
(mostly a work of integration) of this size. I recommend to keep the management
as flexible as possible and to focus on agility.

\paragraph{} A management of this kind can not work without teammates really
involved in the project.

\paragraph{} When reaching a milestone, a teammate shall assess its work by
writing an e-mail to the whole team. When a milestone is reached, a member of
the other group will check the work and write a feedback to the authors. The
reviewer should not correct a defect (even if this is typo or a minor mistake)
and let the author do the rework task and acknowledge the reviewer when done.

\subsection{Specifications and requirements drafting}

\paragraph{} Both groups will work on this task. Since it's the beginning of the
project, both groups should work on the same document and share their revisions
as often as possible in order to check that we're all thinking the same way.

\paragraph{} Video meetings are too complex to set up to be efficient for this
task and must be reserved for the validation of major design decisions. We
should use instant messengers and collaborative work tools (Versioning tools or
Google docs, for instance).

\subsection{Construction phase}

\paragraph{} When the broad outlines of the requirements will appear clearly,
it may not be hard to divide the work in tasks that won't conflict with other's
work. Most of the time, two or three groups can work simultaneously on different
features without real trouble.

\paragraph{} Groups can choose the features they want to work on, the project
manager shall intercede in this repartition if this repartition of tasks does
not take place smoothly.

\paragraph{} A group shall never ship code that is not duly tested and asserted
to compile or run. A code that breaks other's work is a pain for all!

\subsection{Test phase}

\paragraph{} Tests during construction must be performed by the authors of the
work, and then by an other group who will test the features and review the code
(not exactly "test" the code).

\paragraph{} Users tests must be moderated by all the teammates and reviews must
be compiled and sent to the group who will work on the issues collected.

\paragraph{} A work review is not a really heavy procedure. If someone has
trouble when trying to review some work, it highlights understanding issues and
denotes that the work is too complex or that the reviewer has not clearly
understood the goals targeted by the work he is reviewing.

\paragraph{} In order to keep reviews easy and effective, a review of about 15
minutes must be performed for about each 4 hours of work.

\subsection{Control and QA checks}

\paragraph{} In order to be effective, we should discuss with only the minimum
communication channel/tools. I think that we should use e-mails and a discussion
group where we will centralize most of our communication. By the way, it does
not include conversation between teammates working together on a task who are
using instant messaging, but any important information must be reported in an
e-mail.

\paragraph{} When someone worked on a task, he must send an e-mail to the group
and tell what he did and provide a few explanations about his work allowing the
others (and especially the reviewer) to understand the work.

\paragraph{} The Quality Manager will perform a QA review and, with the project
manager, will validate a reviewed and corrected work. The work of the QM at this
level can be seen as a kind of second review.

\newpage
\section{Supporting tools we will use}

\subsection{Communication}

\paragraph{} We'll set-up a mailing-list (probably using Google Group, but any
suggestion here is welcome). Excluding video meetings which have a particular
state, the mailing-list is the only official communication platform.

\paragraph{} Video meetings shall be done with Skype and/or Google talk
according to the preference of the majority and technical limitations. In France,
Skype is stupidly considered as potentially harmful because when it had been
launched, it produced a lot of encrypted packets over universities networks.
In a nutshell, Skype can be blocked by a firewall.

\paragraph{} Teammates are free to work and discuss with the tools they want to
use. But these discussions won't be considered as part of the project. For
instance, reviews discussions must most of the time take place via the
mailing-list.

\subsection{Version control}

\paragraph{} I would like to use git and github. It's a Distributed Concurrent
Version System widely used. If you don't know git yet, you'll probably need to
in a near future, so why not discover this tool right now ? The great book "Pro
Git" is available for free (and legally) on the internet.

\paragraph{} Github is a source forge we can use for our project. It provides a
wiki and a ticketing system.

\paragraph{} Documentation can be written in Latex on a git repository or Google
Doc. The later is recommended for the sake of simplicity.

\subsection{Management}

\paragraph{} I think that most of the features a management software (Microsoft
Project, Redmine, etc) would provide are already available with other tools we
will use, except for a Gantt-planing tool, which I (as Project Manager) don't
plan to use.

\newpage
\section{Conclusion}

\paragraph{} This document will probably be updated with more accurate
information over the time. Feel free to ask any question about its content
before the project really starts.

\paragraph{} Some sentences of paragraphs may be unclear because of my skills in
English. Don't hesitate to ask me to rephrase or give you more explanations.

\paragraph{} If you don't agree with parts of the document, I think that we can
discuss the issue together by e-mail or during a video meeting.

\paragraph{} I wrote this document using Latex, I can put the sources on a
repository if you want to rework parts of it.



\end{document}
