\section{Detailed description of the project}

\subsection{Summary}

\paragraph{} This section reuses and enforces the Introduction document redacted
and sent to the German teammates on week 42.

\paragraph{} We would like to create a website offering guides, tips and
information about student exchanges, written by students and for other
students.

\paragraph{} Not much details about the structure of the website are provided,
but we want the website to clearly separate content about an university from
an other. We will therefore structure and present the information by countries
and universities.

\paragraph{} The project is born at the initiative of the department, seeing
each year a demand from students for a tool provided by the school gathering
information about exchanges that are out of the scope of the school's
Direction for International Relationship (Direction des relations
internationales, a.k.a. DRI).

\paragraph{} The tool is addressed to students of the school, that's why we want
them to be the first contributors. For this matter, we can think about existing
services and web applications involving users in the content creation process:
wikis, blog, bulletin boards, etc.

\subsection{Goals and expected achievements}

\paragraph{} Our first task will be to write out a proposal of the website
structure and features. This proposal will gather demands from students and
teachers and be validated by Mme Rumpler, for the direction of the IT
department of the INSA.

\paragraph{} We are quite free to propose anything we think can fit in the
project, this phase of the project is really open. Moreover, even if the project
will eventually be destined to INSA student's, we would be please if the
final product is good enough to be deployed in other schools.

\paragraph{} We hope to be able to turn the website online by the end of the
year, and provide a stable version during January.

\paragraph{} The website will be deployed on a server hosted by the school IT
infrastructure.

\subsection{Technical requirements}

\paragraph{} Users will not have access to the website's backend, whichever it
is (access to the hosting server for static pages update or administration
interface of a CMS in the browser). Features of the website will allow users to
update selected parts of the content from the website's "public" pages.

\paragraph{} We must host the website on a server provided by our local student
club. It will probably be a classic LAMP stack (Linux/Apache/Mysql/PHP). We will
not be able to choose to work with an other language (Ruby, python, ...).

\paragraph{} Access to the content of the website must be restricted to
students. A Centralized Authentication System (CAS) had been deployed on our
campus. CAS is a standard technology allowing to authenticate user on many
web services using one centralized couple login/password by user. We must use
this platform to grant access to users on the website.

\subsection{Recommendations}

\paragraph{} It has been explicitly asked not to implement a rating or reviews
system, as the teachers don't want the website to contain negative comments from
students.

\paragraph{} The following point is not mandatory but is a high recommendation
made in order to be able to complete the project at defined deadlines. We should
start by looking to existing Content Management Systems (CMSs) in order to avoid
the construction of a such tool outside of the scope of the project.
