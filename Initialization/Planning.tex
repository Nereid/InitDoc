\section{Forecast planning}

\paragraph{} The following forecast planning is presented in order to describe
the project execution. This planning will change and become more accurate over
the time. This planning doesn't describe tasks but work packages. A forecast
planning with too much detail shall be thrown away because the lack of details
on the content of the final product.

\paragraph{} The project will last 10 weeks, including the two weeks of
Christmas Holidays in France. This planning starts at week 44.

\paragraph{} The following list is an estimation of the duration of each project
steps, taking into account a student agenda.

\begin{itemize}
	\item Project set-up: 1 week
	\item Website specifications: 2 week
	\item CMSs benchmarks and tests: 0.5 week
	\item Set-up of the development environment: 0.5 week
	\item Construction (Integration, code and tests): 3 weeks
	\item Users tests and validation: 1 week
	\item Release: 0 week (End of the project)
	\item Total: 8 weeks
\end{itemize}

\paragraph{} I think that holidays are meant for a good reason, we shall not
need to work too much during the last two weeks of the year if we're not too
late.

\paragraph{} The following list gives a hint about expected deadlines (consider
the end of the day). Dates are in format Month/Day.

\begin{itemize}
	\item Project set-up: Friday 11/4
	\item Website specifications: 11/22
	\item CMSs benchmarks and tests: 11/22
	\item Set-up of the development environment: 11/25
	\item Construction (Integration, code and tests): 12/17
	\item Users tests and validation: 1/4
	\item Release: 1/9
\end{itemize}

\paragraph{} As you may have noticed, write website specifications and
benchmarking CMSs share the same deadline. These two activities must be
performed simultaneously. User tests and validation requires the team to collect
feedback from other students and test the website. I think we can do it during
the holidays, in order to preserve a few days at the end of the project to
prepare the final presentation or work on various details.
