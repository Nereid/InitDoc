\section{Management methods}

\paragraph{} Classical management method don't fit well with a web project
(mostly a work of integration) of this size. I recommend to keep the management
as flexible as possible and to focus on agility.

\paragraph{} A management of this kind can not work without teammates really
involved in the project.

\paragraph{} When reaching a milestone, a teammate shall assess its work by
writing an e-mail to the whole team. When a milestone is reached, a member of
the other group will check the work and write a feedback to the authors. The
reviewer should not correct a defect (even if this is typo or a minor mistake)
and let the author do the rework task and acknowledge the reviewer when done.

\subsection{Specifications and requirements drafting}

\paragraph{} Both groups will work on this task. Since it's the beginning of the
project, both groups should work on the same document and share their revisions
as often as possible in order to check that we're all thinking the same way.

\paragraph{} Video meetings are too complex to set up to be efficient for this
task and must be reserved for the validation of major design decisions. We
should use instant messengers and collaborative work tools (Versioning tools or
Google docs, for instance).

\subsection{Construction phase}

\paragraph{} When the broad outlines of the requirements will appear clearly,
it may not be hard to divide the work in tasks that won't conflict with other's
work. Most of the time, two or three groups can work simultaneously on different
features without real trouble.

\paragraph{} Groups can choose the features they want to work on, the project
manager shall intercede in this repartition if this repartition of tasks does
not take place smoothly.

\paragraph{} A group shall never ship code that is not duly tested and asserted
to compile or run. A code that breaks other's work is a pain for all!

\subsection{Test phase}

\paragraph{} Tests during construction must be performed by the authors of the
work, and then by an other group who will test the features and review the code
(not exactly "test" the code).

\paragraph{} Users tests must be moderated by all the teammates and reviews must
be compiled and sent to the group who will work on the issues collected.

\paragraph{} A work review is not a really heavy procedure. If someone has
trouble when trying to review some work, it highlights understanding issues and
denotes that the work is too complex or that the reviewer has not clearly
understood the goals targeted by the work he is reviewing.

\paragraph{} In order to keep reviews easy and effective, a review of about 15
minutes must be performed for about each 4 hours of work.

\subsection{Control and QA checks}

\paragraph{} In order to be effective, we should discuss with only the minimum
communication channel/tools. I think that we should use e-mails and a discussion
group where we will centralize most of our communication. By the way, it does
not include conversation between teammates working together on a task who are
using instant messaging, but any important information must be reported in an
e-mail.

\paragraph{} When someone worked on a task, he must send an e-mail to the group
and tell what he did and provide a few explanations about his work allowing the
others (and especially the reviewer) to understand the work.

\paragraph{} The Quality Manager will perform a QA review and, with the project
manager, will validate a reviewed and corrected work. The work of the QM at this
level can be seen as a kind of second review.
